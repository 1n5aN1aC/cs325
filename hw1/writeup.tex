\documentclass[letterpaper,10pt,titlepage,fleqn]{article}

%example of setting the fleqn parameter to the article class -- the below sets the offset from flush left (fl)
\setlength{\mathindent}{1cm}

\usepackage{graphicx}                                        
\usepackage{amssymb}                                         
\usepackage{amsmath}                                         
\usepackage{amsthm} 
\usepackage{esint}
\usepackage{nopageno}
\usepackage{booktabs}                            
\usepackage{alltt}                                           
\usepackage{float}
\usepackage{color}
\usepackage{fancyhdr}
\usepackage{url}
\usepackage{balance}
\usepackage[TABBOTCAP, tight]{subfigure}
\usepackage{enumitem}
\usepackage{pstricks, pst-node}
%the following sets the geometry of the page
\usepackage{geometry}
\geometry{textheight=9in, textwidth=6.5in}
\pagestyle{fancy}
% random comment
\newcommand{\cred}[1]{{\color{red}#1}}
\newcommand{\cblue}[1]{{\color{blue}#1}}
\usepackage{hyperref}
\usepackage{textcomp}
\usepackage{listings}
\def\name{Best CS325 Group}
%% The following metadata will show up in the PDF properties
\hypersetup{
  colorlinks = true,
  urlcolor = black,
  pdfauthor = {\name},
  pdfkeywords = {cs311 ``operating systems'' files filesystem I/O},
  pdftitle = {Pertinent Information},
  pdfsubject = {Virtual Reality Lab},
  pdfpagemode = UseNone
}

\parindent = 0.0 in
\parskip = 0.2 in
\fboxsep=5mm%padding thickness
\fboxrule=4pt%border thickness

\begin{document}
\lstset{language=Python} 

\title{Programming Assignment #1 - CS325}
\date{January 29, 2014}
\maketitle
%to remove page numbers, set the page style to empty

\section*{Pseudocode}
\hrule
\begin{centering}

\underline{\large{\textbf{Brute Force:}}}\\
\end{centering}
\begin{lstlisting}
%Brute force pseudocode goes here
\end{lstlisting}

\begin{centering}
\underline{\large{\textbf{Naive Divide and Conquer:}}}\\
\end{centering}
\begin{lstlisting}
%Naive divide and conquer pseudocode goes here
\end{lstlisting}

\begin{centering}
\underline{\large{\textbf{Merge and Count:}}}\\
\end{centering}
\begin{lstlisting}
%Merge and count  pseudocode goes here
\end{lstlisting}

\section*{Correctness Proof}
\hrule
%Correctness proof goes here

\section*{Asymptotic Analysis of Run Time}
\hrule
\begin{centering}
\underline{\large{\textbf{Brute Force:}}}\\
\end{centering}
It has two for loops of size n duh

\begin{centering}
\underline{\large{\textbf{Naive Divide and Conquer:}}}\\
\end{centering}
T(n) = this class is difficult

\begin{centering}
\underline{\large{\textbf{Merge and Count:}}}\\
\end{centering}
T(n) = wow such recursion


\section*{Testing}
\hrule
Results and shit

\section*{Extrapolation and Interpretation}
\hrule
\begin{itemize}
\item Largest input item solvable in an hour:
\item Slope of lines in log-log plot:
\item Discrepancy between actual and asymptotic:
\end{itemize}


\end{document}
