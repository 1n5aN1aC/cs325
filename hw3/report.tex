\documentclass[letterpaper,10pt,titlepage,fleqn]{article}

%example of setting the fleqn parameter to the article class -- the below sets the offset from flush left (fl)
\setlength{\mathindent}{1cm}

\usepackage{graphicx}                                        

\usepackage{amssymb}                                         
\usepackage{amsmath}                                         
\usepackage{amsthm} 
\usepackage{esint}
\usepackage{nopageno}
\usepackage{booktabs}
                                         

\usepackage{alltt}                                           
\usepackage{float}
\usepackage{color}
\usepackage{fancyhdr}
\usepackage{url}

\usepackage{balance}
\usepackage[TABBOTCAP, tight]{subfigure}
\usepackage{enumitem}

\usepackage{pstricks, pst-node}

%the following sets the geometry of the page
\usepackage{geometry}
\geometry{textheight=9in, textwidth=6.5in}
\pagestyle{fancy}
% random comment

\newcommand{\cred}[1]{{\color{red}#1}}
\newcommand{\cblue}[1]{{\color{blue}#1}}

\usepackage{hyperref}

\usepackage{textcomp}
\usepackage{listings}

\def\name{CS325 Group}

%% The following metadata will show up in the PDF properties
\hypersetup{
  colorlinks = true,
  urlcolor = black,
  pdfauthor = {\name},
  pdfkeywords = {cs311 ``operating systems'' files filesystem I/O},
  pdftitle = {CS 311 Project},
  pdfsubject = {CS 311 Project},
  pdfpagemode = UseNone
}

\parindent = 0.0 in
\parskip = 0.2 in

\begin{document}
\lstset{language=Python} 

\title{Travelling Salesman Problem - CS325}

\author{
	Joshua Villwock \and
	Jaron Thatcher \and
	Ryan Phillips
}

\date{March 14, 2014}
\maketitle


%to remove page numbers, set the page style to empty

\section*{Our Solution}
\hrule

\begin{Large}
\textbf{Introduction} \\
\end{Large}
To prepare for solving the given test cases within the testing window, we wrote a suite of TSP-related functions/programs. These can be divided into four categories: analysis, solvers, optimizers, and a client-server system. These elements were then used together in a strategy that was actively run by all participants during the testing window. \\

\begin{Large}
\textbf{Analysis} \\
\end{Large}
In order to determine how effective each approach was we generated and ran randomized test cases over a range of inputs. Matplotlib was then used to generate a number of plots including the city sets, particular routes of interest, route length vs. time, and route length vs. N. \\

\begin{Large}
\textbf{Solvers} \\
\end{Large}
To get an initial solution, we wrote a couple of basic TSP-solvers. These included the following types of algorithms: exact, greedy, and minimum spanning tree.

“Exact” simply tried to every possible permutation on the city set and returned the one with the shortest route length. This wasn’t practical for any of the given test set, but it is a good approach for a small set of cities (like less than 10). This algorithm has a runtime of O(n!), where n is the number of cities.

The greedy algorithm starts at a specified starting city and then picks the closest city to it as the next stop in its route. This new city is removed from the set of possible route destinations, and the process is repeated as long any unvisited cities remained. This algorithm has a runtime of O(n*n). A variation of this which gave consistently better results (at the expense of another factor of n).

The tsp solver began by constructing a graph representation of the cities by calculating the distances between every pairs of nodes. It then picked the smallest distance from the list, and added that edge and the two cities that connected to it to the graph. Next... \\

\begin{Large}
\textbf{Optimizers} \\
\end{Large}
These functions took as input a set of cities and a valid route and would try making changes, and then continually test to see if any of the changes yielded a shorter route.

Reverse Optimization: Given a particular section length, this function reverses the order of every possible sub-route of this size.

City Swap Optimization: Takes 2 or more cities and swaps their positions in the route. This function had a couple of different flavors, including a randomized and a systematic method. The second flavor attempted every possible swap and the first just picked from a random subset of these possibilities.

City Transpose Optimization: For every city in the route, try shifting it to every other possible position in the route. There was also a variation of this function that performed this same operation with not just a single city, but with a subroute.

Exact Section Optimization: Given a particular subroute, it tries every possible permutation to see which is optimal. This uses the same approach as the exact TSP solver, so the same limits apply. We only tried it with subroute lengths less than ten. \\






\end{document}
